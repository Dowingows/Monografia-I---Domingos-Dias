\documentclass[../main.tex]{subfiles}
 
\begin{document}

% ----------------------------------------------------------
% INTRODUÇÂO
% ----------------------------------------------------------

\chapter{Introdução}

O mercado de energia em diversos países está passando por mudanças, basicamente em função da desregulamentação do mercado e do surgimento de mecanismos judiciais e administrativos de defesa do consumidor~\cite{ibanez2006antecedents}. No Brasil, a necessidade de mudanças no mercado de concessionárias de energia elétrica se tornou mais evidente nos últimos anos, principalmente devido ao surgimento de ações judiciais contra essas concessionárias. Somente no estado do Rio de Janeiro, aproximadamente 30 mil novas ações foram registradas a cada mês contra empresas de distribuição de energia, o que gera uma despesa aproximada de 100 milhões de reais~\cite{almeida2014propensao}.


Quando se trata de uma concessionária de energia elétrica, cujo objetivo principal é baseado na prestação de serviços ao consumidor final, há ainda maiores dificuldades na eliminação de falhas deste serviço~\cite{johnston1999service}. Além disso, no Brasil, cada concessionária opera em um domínio extenso, muitas vezes abrangendo um estado completo da federação. Além de fornecer e administrar o suprimento de energia elétrica ao consumidor, ainda há uma gama de funções burocráticas ligadas a essa prestação de serviços que podem gerar insatisfação do cliente, resultando em custos judiciais para as concessionárias, como serviços: atendimento ao cliente; interrupção de energia; manutenção de Equipamento; entre outros.

Nesse contexto, é necessário que as empresas aprendam a corrigir suas falhas de maneira rápida e eficiente, não apenas para evitar custos legais, mas também se preocupando com a reputação e a qualidade do serviço prestado. Além disso, entender o cliente (ou grupo de clientes), entender os possíveis fatores correlatos com a questão dos processos judiciais, facilita a tomada de decisões da empresa e decidir qual setor deve atuar de forma adequada para prevenir e/ou erradicar custos desnecessários.


Entender o consumidor de energia e prever seu comportamento pode contribuir significativamente para reduzir os processos judiciais, aumentando a qualidade e a eficiência da empresa, detectando o problema nos estágios iniciais e tomando as medidas corretas de tratamento. Supõe-se que os dados históricos do relacionamento com o cliente da empresa podem ser um importante ponto de partida para a criação de mecanismos inteligentes de mineração e identificação de clientes que possam abrir desafios legais para vários assuntos contra a empresa.


% ----------------------------------------------------------
% OBJETIVO                           
% ----------------------------------------------------------
\section{Objetivo}

Este trabalho tem como objetivo desenvolver uma metodologia para prever processos judiciais com a utilização de algorítimos de séries temporais.
% ----------------------------------------------------------
% OBJETIVOS ESPECÍFICOS                           
% ----------------------------------------------------------
\subsection{Objetivos Específicos}
\begin{itemize}
    \item Investigar e aplicar métodos de engenharia de características para dados temporais;
	\item Analisar e desenvolver abordagem baseada em \textit{meta-learning} para a estimação automática de algoritmo de aprendizagem de máquina;
	\item Investigar métricas de avaliação de modelos de previsão em séries temporais;
	\item Realizar experimentos com modelo proposto em bases de dados privadas;
\end{itemize}

% ----------------------------------------------------------
% JUSTIFICATIVA                           
% ----------------------------------------------------------
\section{Justificativa}

A análise temporal do comportamento do consumidor de energia elétrica pode ser um importante indicador para mensurar o grau de satisfação com o serviço prestado pela empresa fornecedora. O grau de satisfação pode ser um indicativo essencial para identificar novos casos de processos judiciais. De posse da solução proposta por este trabalho, uma concessionária de energia elétrica terá meios de intervir junto aos clientes insatisfeitos, afim de evitar uma nova ação judicial.


\end{document}

