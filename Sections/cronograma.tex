\documentclass[../main.tex]{subfiles}
 
\begin{document}

\chapter{Cronograma}
Nesta seção são apresentadas as atividades a serem realizadas e suas respectivas previsões, conforme mostra a Tabela \ref{CRONOGRAMA}.

\begin{enumerate}

\item[ a) ] Levantamento bibliográfico.
\item[ b) ] Aquisição dos dados.
\item[ c) ] Pesquisa e implementação das técnicas de pré-processamento.
\item[ d) ] Implementação e aplicação da metodologia de transformação dos dados.
\item[ e) ] Estudo e aplicação de algoritmos de séries temporais.
\item[ f) ] Avaliação da solução proposta.
\item[ g) ] Escrita da monografia.
\item[ h) ] Defesa.

\end{enumerate}

\begin{table}
\centering
\caption{Cronograma de Atividades}
\label{CRONOGRAMA}
\begin{tabular}{|c|c|c|c|c|c|c|}
\hline
Atividades & Agosto & Setembro & Outubro & Novembro & Dezembro\\\hline
a) & x & x &   &   &   \\\hline
b) & x &   &   &   &   \\\hline
c) &   & x &   &   &   \\\hline
d) &   & x &   &   &   \\\hline
e) &   & x & x & x &   \\\hline
f) &   &   &   & x &   \\\hline
g) &   &   & x & x &   \\\hline
h) &   &   &   &   & x \\\hline
\end{tabular}
\end{table}


\end{document}