\documentclass[../main.tex]{subfiles}
 
\begin{document}

% ---------------------------------------------------
% METODOLOGIA                                        
% ---------------------------------------------------
\chapter{Trabalhos relacionados}

Apesar de ser um ramo com enorme apelo social e com muita preocupação parte das concessionárias de energia, metodologias computacionais baseadas em inteligência computacional ainda são muito escassas, principalmente um método completamente automático de predição de processos judiciais. Porém, existem alguns trabalho na literatura que tratam de satisfação de clientes, análise de perfil de consumidor, e rotatividade de consumidor em empresas. Este trabalhos serão listados como relacionados, mesmo que não sejam possíveis comparação com o método proposto, foram fundamentais para o embasamento deste trabalho. 

A seguir serão apresentados  trabalhos sobre a análise do comportamento judicial do cliente e a tomada de decisões em diferentes empresas. O trabalho desenvolvido por \citeonline{marchetti2004avaliaccao} propõe um modelo de avaliação da satisfação do consumidor baseado no modelo de equações estruturais. Estes modelos utilizam variáveis latentes, isto é, características que, não podendo ser observadas diretamente, são representadas ou medidas a partir de outras variáveis, isto é, pelos indicadores. Para avaliação do modelo, foram coletadas 300 entrevistas em cada uma das 62 concessionárias de energia elétrica brasileiras, com amostragem probabilística por área, o que completou uma amostra de 18.600 casos válidos. Os resultados obtidos indicam um modelo de grande validade e consistência interna, bem como flexível para mensurar a satisfação em diversas situações: para diferentes portes de empresa, regiões do país etc.

Melhorar a qualidade dos serviços é o principal método para diminuir as ações legais dos clientes. Assim, a satisfação do cliente pode ser uma das causas do risco de ações judiciais. Para o setor elétrico brasileiro, a satisfação dos clientes tem sido modelada por meio de indicadores de qualidade de serviço definidos por um processo de entrevista~\citeonline{marchetti2004avaliaccao}. Assim, observou-se, por meio da modelagem das intenções do cliente, que a qualidade das transações é uma característica dominante na satisfação e fidelidade do cliente~\citeonline{afonso2011relationship}.


A preocupação pela melhoria dos serviços prestados é comum em todas as empresas como uma forma de entender melhor o cliente, se tornar mais competitiva e diminuir a litigância por reclamação. Modelar a satisfação tem sido objetivo de pesquisas como em \citeonline{marchetti2004avaliaccao} onde a satisfação do consumidor de energia elétrica do Brasil é medido através da construção de características sobre indicadores de qualidade colhidos via entrevista. \citeonline{afonso2011relationship} também propõe um modelo para medir intenções comportamentais dos clientes em uma determinada empresa tendo como resultados que a qualidade das transações surge como uma característica dominante, com um efeito forte e positivo e que não há efeitos significativos dos custos de mudança na satisfação e no comprometimento dos clientes.


A relação entre os eventos anteriores, que poderiam ser tratados individualmente, com a qualidade e fidelidade do relacionamento é o objetivo do trabalho proposto por \citeonline{francisco2011evaluation} em uma empresa de telefonia móvel. O estudo foi realizado com um quadro quantitativo não probabilístico e contou com o método hipotético-dedutivo. Os dados foram obtidos de uma pesquisa com 493 usuários de telefones celulares e analisados por meio de uma equação estrutural e regressão logística. O estudo confirmou (direta e indiretamente) as premissas de um efeito positivo e significativo entre satisfação, confiança e comprometimento e seus antecedentes. Satisfação e lealdade foram elementos distintivos entre os grupos com menor propensão a mudar de operador.

De acordo com~\citeonline{siu2013roles}, os clientes reclamam porque querem ser tratados de forma justa pela empresa quando ocorre uma falha no serviço. Este estudo investiga o papel da justiça na retenção de clientes que tiveram experiências falhas com o serviço de restaurante. Como resultado, os autores confirmam a relação de justiça com a satisfação anterior e posterior, queixa aos clientes.


A crescente demanda por consumo financeiro dos clientes intensifica ainda mais a competição entre os bancos comerciais. Para aumentar seus lucros para operações contínuas e aumentar a competitividade do núcleo, os bancos comerciais devem evitar a perda de clientes enquanto adquirem novos clientes. \citeonline{He2014} prevêem a rotatividade de clientes de banco comercial com base no modelo SVM e usa o método de amostragem aleatória para melhorar o modelo SVM. Quando a relação é de 1:10 (clientes com rotatividade:clientes sem rotatividade), as taxas de recall, precisão e acurácia, respectivamente, são de 56,09\%, 29,50\% e 98,39\%.


Métodos preditivos foram construídos com uma meta correlacionada de manutenção do cliente. A previsão de mudança de serviço é um problema comum, não apenas em empresas de distribuição de energia, mas também em muitas outras empresas em outros setores, como telecomunicações~\citeonline{Amin2017,Amin2019}, bancos~\citeonline{He2014,Keramati2016} e jogos~\citeonline{Milosevic2017,Kim2017}.


Com base no que é observado na literatura, e a importância de prevenir ações judiciais de clientes com base em suas características, o método proposto neste trabalho tem como premissa a construção de um modelo preditivo. Para tanto, utiliza-se com características extraídas e geradas em dados de relacionamento temporal de consumidores de uma empresa de energia elétrica. O objetivo é medir, de antemão, mudanças grau de insatisfação do cliente, assim, fornecer informações preventivas para que gerentes e técnicos lidem da melhor maneira possível para resolver o problema.

%O objetivo é promover um ambiente preventivo que possa corroborar com a redução de ações judiciais.

\end{document}